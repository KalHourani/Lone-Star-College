\documentclass[a4paper]{article}

%% Language and font encodings
\usepackage[english]{babel}
\usepackage[utf8x]{inputenc}
\usepackage[T1]{fontenc}

%% Sets page size and margins
\usepackage[a4paper,top=3cm,bottom=2cm,left=3cm,right=3cm,marginparwidth=1.75cm]{geometry}

%% Useful packages
\usepackage{amsmath}
\usepackage{graphicx}
\usepackage[colorinlistoftodos]{todonotes}
\usepackage[colorlinks=true, allcolors=blue]{hyperref}
\usepackage{listings}
\usepackage{minted}
\usepackage{tikz-qtree}

%% Title
\title{Traversal of Binary Trees}
\author{Khalid Hourani}

\begin{document}
There are four ways to traverse a binary tree. They are


\section{Pre-order Traversal}
To traverse a binary tree with a Pre-Order Traversal, begin at the root of the tree:
\begin{enumerate}
	\item Check if the current node is empty. 
	\item Visit the current node.
	\item Traverse the left subtree, recursively calling the function on the visited nodes.
	\item Traverse the right subtree, recursively calling the function on the visited nodes.
\end{enumerate}
\section{Post-order Traversal}
To traverse a binary tree with a Post-Order Traversal, begin at the root of the tree:
\begin{enumerate}
	\item Check if the current node is empty. 
	\item Traverse the left subtree, recursively calling the function on the visited nodes.
	\item Traverse the right subtree, recursively calling the function on the visited nodes.
	\item Visit the current node. 
\end{enumerate}
\section{In-order Traversal}
\section{Euler Tour Traversal}

\begin{figure}[h]
 \includegraphics[scale=0.25]{Tree.png}
\end{figure}

\end{document}
