\documentclass[a4paper]{article}

%% Language and font encodings
\usepackage[english]{babel}
\usepackage[utf8x]{inputenc}
\usepackage[T1]{fontenc}

%% Sets page size and margins
\usepackage[a4paper,top=3cm,bottom=2cm,left=3cm,right=3cm,marginparwidth=1.75cm]{geometry}

%% Useful packages
\usepackage{amsmath}
\usepackage{graphicx}
\usepackage[colorinlistoftodos]{todonotes}
\usepackage[colorlinks=true, allcolors=blue]{hyperref}
\usepackage{listings}
\usepackage{minted}
\usepackage{tikz-qtree}

%% Title
\title{Sorting Algorithms}
\author{Khalid Hourani}

\begin{document}
We consider each sort on a 0-indexed array $A$ of $n$ elements.
\begin{enumerate}
 \item Selection Sort
 
On the $i^{\text{th}}$ iteration (beginning with 0), we pass through the array from $A_i$ to $A_{n-1}$ to find the smallest value. Thus, the number of operations is given by \[(n-1)+(n-2)+\hdots+2+1=\frac{n(n-1)}{2}\] A selection sort is therefore $O(n^2)$.
 
 \item Insertion Sort
 
On the $i^{\text{th}}$ iteration (beginning with 0), we place $A_{i+1}$ in the correct position relative to the first $i+1$ elements. In the \textbf{worst case}, the array will be in reverse-order, and elements $A_0$ through $A_i$ must be shifted one to the right, with $A_{i+1}$ being placed at the beginning of the array. That is a total of \[1+2+\hdots+(n-2)+(n-1)=\frac{n(n-1)}{2}\] operations. An insertion sort is therefore $O(n^2)$ \textbf{in the worst case}.

In the \textbf{best case}, the array is already sorted and we only need to pass through the array once, in which case the sort is $O(n)$. 

 \item Merge Sort
\end{enumerate}

\end{document}
