\documentclass[a4paper]{article}

%% Language and font encodings
\usepackage[english]{babel}
\usepackage[utf8x]{inputenc}
\usepackage[T1]{fontenc}
\usepackage{listings}
\usepackage{minted}

%% Sets page size and margins
\usepackage[a4paper,top=3cm,bottom=2cm,left=3cm,right=3cm,marginparwidth=1.75cm]{geometry}

%% Useful packages
\usepackage{amsmath}
\usepackage{graphicx}
\usepackage[colorinlistoftodos]{todonotes}
\usepackage[colorlinks=true, allcolors=blue]{hyperref}

\title{Big O Notation}
\author{Khalid Hourani}

\begin{document}
\maketitle

We say a function $f(n)$ is $O(g(n))$, written \begin{align*}f(n) &\text{ is } O(g(n))\text{ or }\\f(n)&=O(g(n))\text{ or }\\f(n)&\in O(g(n))\end{align*} if and only if there exists a real constant $c>0$ and an integer constant $m\geq1$ such that \[f(n)\leq cg(n) \text{ for all }n\geq m\] In other words, $f(n)$ is $O(g(n))$ when it is bounded from above by $g(n)$. In particular, we use this for analyzing an algorithm to discuss its runtime and memory space. For example, the following algorithm for checking if a number is prime runs in $O(\sqrt{n})$ time:

\begin{minted}{C++}
int isPrime(int n)
{
	for (int i = 2; i * i < n; i++)
	{
		if (n % i == 0)
		{
			return false;
		}
	return true;
	}
}
\end{minted}

because this algorithm will perform an operation for each integer less than $sqrt(n)$. 
\end{document}
