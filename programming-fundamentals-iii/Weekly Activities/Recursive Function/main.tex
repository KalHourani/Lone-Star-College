\documentclass[a4paper]{article}

%% Language and font encodings
\usepackage[english]{babel}
\usepackage[utf8x]{inputenc}
\usepackage[T1]{fontenc}
\usepackage{listings}
\usepackage{minted}

%% Sets page size and margins
\usepackage[a4paper,top=3cm,bottom=2cm,left=3cm,right=3cm,marginparwidth=1.75cm]{geometry}

%% Useful packages
\usepackage{amsmath}
\usepackage{graphicx}
\usepackage[colorinlistoftodos]{todonotes}
\usepackage[colorlinks=true, allcolors=blue]{hyperref}

\title{Recursively Convert a String to an Integer}
\author{Khalid Hourani}

\begin{document}
\maketitle
Let $S$ be a 0-indexed string of $n$ digit characters and let $S_i$ denote the $i^{\text{th}}$ character in $S$. The integer value of $S$, call it $a$, is given by \[a=10^{n-1}S_0+10^{n-2}S_1+\cdots+10S_{n-2}+S_{n-1}\] Or, equivalently, \[a=S_{n-1}+10S_{n-2}+\cdots+10^{n-2}S_1+10^{n-1}S_0\]

Define the function $C$ which takes a character digit and converts it to an integer. Then the function $F$ can be recursively defined to convert a string to an integer: \[F(S)=\begin{cases}C(S_0) & n = 1\\C(S_{n-1}) + 10F(S_{[0:n-1]}) & n > 1\end{cases}\] where $S_{[a:b]}$ denotes the substring of $S$ from index $a$ to index $b-1$.

We demonstrate this function on the string ``12345'':\begin{align*}F(``12345\text{''})&=C(`5\text{'})+10F(``1234\text{''})\\&=5+10(C(`4\text{'})+F(``123\text{''}))
\\&=5+10(4+10(C(`3\text{'})+10F(``12\text{''}))
\\&=5+10(4+10(3+10(C(`2\text{'})+10F(``1\text{''})))
\\&=5+10\cdot4+100(3+10(2+10C(`1\text{'})))
\\&=5+10\cdot4+100\cdot3+1000(2+10(1))
\\&=5+10\cdot4+100\cdot3+1000\cdot2+10000\cdot1
\\&=12345\end{align*}


In C++, we define the function $C$ as \textbf{chtoi}:

\begin{minted}{C++}
int chtoi(char c)
{
  return c - '0';
}
\end{minted}

This allows us to define the function $F$ as \textbf{stoi} in the following way:

\begin{minted}{C++}
int stoi(string c, int n)
{
  if (n == 1)
  {
    return chtoi(c[0]);
  }
  else
  {
    return chtoi(c[0]) + 10 * stoi(c.substr(0, n - 1));
  }
}
\end{minted}

We pass the length of the string, $n$, as a parameter of \textbf{stoi} in order to avoid recalculating the length of each substring. 
\end{document}
