\documentclass[a4paper]{article}

%% Language and font encodings
\usepackage[english]{babel}
\usepackage[utf8x]{inputenc}
\usepackage[T1]{fontenc}
\usepackage{listings}
\usepackage{minted}

%% Sets page size and margins
\usepackage[a4paper,top=3cm,bottom=2cm,left=3cm,right=3cm,marginparwidth=1.75cm]{geometry}

%% Useful packages
\usepackage{amsmath}
\usepackage{graphicx}
\usepackage[colorinlistoftodos]{todonotes}
\usepackage[colorlinks=true, allcolors=blue]{hyperref}

\title{Stacks and Queues}
\author{Khalid Hourani}

\begin{document}
\maketitle

A \textbf{Stack} is an abstract data type that satisfies \textbf{Last-In First-Out} or \textbf{LIFO}. This means that the \textit{last item} placed in the stack is the first one removed. A simple example that helps one remember this is a stack of papers: the most recent paper placed on the stack is the next one that will be removed. 

Conversely, a \textbf{Queue} satisfies \textbf{First-In First-Out}. This means that the first \textit{first item} placed in a queue is the first one removed. A simple example is a line of people at a venue: the person at the front of the line is the next person to get in to the venue. 

A stack has two operations: \textbf{push}, which adds an element to the top of the stack, and \textbf{pop}, which removes the element at the top of the stack.

A queue similarly has two operations: \textbf{enqueue}, which adds an element to the end of the queue, and \textbf{dequeue} which removes the element at the front of the queue.

The primary difference between the two is the order in which elements are removed. A stack satisfies LIFO, whereas a queue satisfies FIFO.

\end{document}
