\documentclass[a4paper]{article}

%% Language and font encodings
\usepackage[english]{babel}
\usepackage[utf8x]{inputenc}
\usepackage[T1]{fontenc}
\usepackage{listings}
\usepackage{minted, color, xcolor}

%% Sets page size and margins
\usepackage[a4paper,top=3cm,bottom=2cm,left=3cm,right=3cm,marginparwidth=1.75cm]{geometry}

%% Useful packages
\usepackage{amsmath}
\usepackage{graphicx}
\usepackage[colorinlistoftodos]{todonotes}
\usepackage[colorlinks=true, allcolors=blue]{hyperref}


\begin{document}
\begin{titlepage}
	\centering
	{\scshape\LARGE Lone Star College \par}
	\vspace{1cm}
	{\scshape\Large Project 1\par}
	\vspace{1.5cm}
	{\huge\bfseries COSC 2336 \par}
	{\huge\bfseries Programming Fundamentals III\par}
	\vspace{0.5cm}
	{\large\bfseries Nina Javaher-Tarash\par}
	\vspace{2cm}
	{\Large\itshape Khalid Hourani\par}
	\vspace{0.5cm}
	{\large March 6, 2017\par}
	\vfill

% Bottom of the page
\end{titlepage}
\clearpage\mbox{}\thispagestyle{empty}\clearpage
\section{Analysis Tools}
\subsection{The Constant Function}
A constant function is any function of the form \[f(n) = c\] where $c$ is some constant. Since any constant function $f(n) = c$ can be written as \[f(n) = c \cdot g(n)\] where $g(n) = 1$ is another constant function, we will typically focus on the constant function $f(n)=1$. The constant function frequently describes the number of steps required for basic operations, like assigning a value to a variable, adding two integers, or accessing an array index. 

\subsection{The Logarithm Function}

  A logarithm function is any function of the form \[f(n)=\log_b(n)\] defined by \[\log_b(x)=y\text{ if and only if } b^y=x\] Typically, while mathematicians write $\log$ to mean $\ln=\log_e$ and engineers and scientists write $\log$ to mean $\log_{10}$, computer scientists write $\log$ to mean $\log_2$. However, any base-b logarithm can be converted to base-a by the formula \[\log_b(n)=\frac{\log_a(n)}{\log_a(b)}\] so the choice of base has little impact in practice. The following four properties of logarithms are notable:
  \begin{enumerate}
  \item $\log_b(mn) = \log_b(m)+\log_b(n)$
  \item $\log_b(m/n) = \log_b(m) - \log_b(n)$
  \item $\log_b(m^n) = n\log_b(m)$
  \item $b^{\log_b(m)} = m$
  \end{enumerate}

  The logarithm function frequently shows up in algorithm analysis in examples like the binary search, which, on a search of $n$ elements, takes at most $\log(n)$ operations.

 \subsection{The Linear Function}

  A linear function is any function of the form \[f(n)=cn\] We similarly typically focus on the linear function $f(n)=n$. This function frequently occurs when we have to perform an operation on each element in a set. For example, a linear search on $n$ elements takes at most $n$ comparison operations.

\subsection{The N-Log-N Function}

  The N-Log-N function is given by \[f(n)=n\log(n)\] This function grows faster than the logarithm function and the linear function, but slower than the quadratic function. This function tends to show up when optimizing a function that runs in quadratic time.
 
 \subsection{The Quadratic Function}

  A quadratic function is any polynomial \[f(n) = an^2+bn+c\] However, we typically focus on the quadratic \[f(n) = n^2\] since \[an^2+bn+c\leq\left(|a|+|b|+|c|\right)n^2\] This function tends to show up when nesting loops. For example, multiplying two $n$-digit numbers using the elementary multiplication algorithm requires $n^2$ operations, since you must iterate through each digit of the first number for each iteration through each digit of the second.

 \subsection{The Cubic Function and Other Polynomials}

  The cubic function is any polynomial \[f(n)=an^3+bn^2+cn+d\] Though, as in the above example, we typically consider \[f(n) = n^3\] A polynomial function of degree $m$ is a function of the form \[f(n)=a_0n^m+a_1n^{m-1}+\cdots+a_{n-1}n+a_m\] Similarly, we typically consider the polynomial \[f(n)=n^m\] These functions show up in cases where there are multiple nested loops. In the case of four nested loops, for example, the function would require approximately $n^4$ operations.

 \subsection{The Exponential Function}

  An exponential function is any function of the form \[f(n)=b^n\] The exponential function can be defined for non-negative integer values recursively as \[b^n=\begin{cases}1 & n = 0\\b\cdot b^{n-1} & n>0\end{cases}\] This definition can be extended to all integers by setting \[b^{-n}=\frac{1}{b^n}\] and to all rational numbers by \[b^{\frac{m}{n}}=\sqrt[n]{b^m}\] Finally, we can extend the definition to all real numbers by defining \[b^r=\lim_{n\to\infty}b^{r_n}\] where $r_n$ is any sequence of rationals such that $r_n\to r$. An equivalent definition is found by setting \[\ln(x)=\int_1^x\frac{1}{t}dt\] and defining $e^x$ as the inverse function of $\ln{x}$. We then set \[b^x=e^{x\ln{b}}\]

  The following three properties are notable:
  \begin{enumerate}
  \item $b^m\cdot b^n=b^{m+n}$
  \item $\frac{b^m}{b^n}=b^{m-n}$
  \item $(b^m)^n=b^{mn}$
  \end{enumerate}

The exponential function frequently shows up when something is repeatedly doubled.

\section{Asymptotes}

\subsection{Asymptotic Notation}

Asymptotic Notation is the use of the following terminology to describe the behavior of an algorithm:
\begin{enumerate}
\item $O(n)$ - \textbf{Big O Notation}
\item $\Omega(n)$ - \textbf{Big Omega Notation}
\item $\Theta(n)$ - \textbf{Big Theta Notation}
\end{enumerate}

We say a function $f(n)$ is $O(g(n))$, written \begin{align*}f(n) &\text{ is } O(g(n))\text{ or }\\f(n)&=O(g(n))\text{ or }\\f(n)&\in O(g(n))\end{align*} if and only if there exists a real constant $c>0$ and an integer constant $m\geq1$ such that \[f(n)\leq cg(n) \text{ for all }n\geq m\]

For example, the function $f(n)=n^2+4n+9$ is $O(n^3)$, since \[n^2+4n+9\leq 1n^3\text{ for } n\geq4\]

Similarly, we say that $f(n)$ is $\Omega(g(n))$, written \begin{align*}f(n) &\text{ is } \Omega(g(n))\text{ or }\\f(n)&=\Omega(g(n))\text{ or }\\f(n)&\in \Omega(g(n))\end{align*} if and only if there exists a real constant $c>0$ and an integer constant $m\geq1$ such that \[f(n)\geq cg(n) \text{ for all }n\geq m\] Equivalently, $f(n)$ is $\Omega(g(n))$ if and only if $g(n)$ is $O(f(n))$. 

For example, the function $f(n)=n^2+4n+9$ is $\Omega(n)$, since \[n^2+4n+9\geq 1n\text{ for }n\geq1\]

Finally, we say that $f$ is $\Theta(g(n))$, written \begin{align*}f(n) &\text{ is } \Theta(g(n))\text{ or }\\f(n)&=\Theta(g(n))\text{ or }\\f(n)&\in \Theta(g(n))\end{align*} if and only if there exist real constants $c'>0$ and $c''>0$ and an integer constant $m\geq1$ such that \[c'g(n) \leq f(n)\leq c''g(n) \text{ for all }n\geq m\] Equivalently, $f(n)=\Theta(g(n))$ if and only if $f(n)=O(g(n))$ \textbf{and} $f(n)=\Omega(g(n))$

For example, the function $f(n)=n^2+4n+9$ is $\Theta(n^2)$, since \[n^2 \leq n^2+4n+9\leq14n^2\text{ for } n\geq1\]

In practice, Big $O$ notation is more frequently used than Big $\Omega$ and Big $\Theta$, since we are often more concerned with a ``worst case'' estimation of the performance and memory space of an algorithm.

\subsection{Asymptotic Analysis}

Asymptotic Analysis is the use of the above three definitions to analyze the speed and memory-efficiency of an algorithm. If a function $f(n)$ is $O(g(n))$, then, for sufficiently large $n$ (hence ``asymptotic''), we know that $f$ is no slower than $g$. For example, if a particular algorithm runs in $O(n^2)$ time, then we know that doubling the size of the input will at most \textbf{quadruple} the runtime. Consider the following algorithm for determining if an integer $n>3$ is prime:

\inputminted[frame=lines,bgcolor=lightgray]{C++}{isPrime.cpp}

We see that this algorithm will perform at most $\sqrt{n}$ operations (when $n$ is a perfect-square or prime), and is therefore $O(\sqrt{n})$. In other words, if we quadruple the input, we will require at most \textbf{double} the number of operations. However, in many cases we will not \textbf{actually} require double the number of operations. Consider the function evaluated on $n=4$. The function will perform one operation before returning false, simply testing divisibility by 2. When we evaluate the function on $16$, it similarly only takes only a single operation. In fact, this best-case scenario (when $n$ is even) shows that the algorithm is $\Omega(1)$. 

Similarly, consider, the following function to recursively evaluate $x^n$ for non-negative integers $n$:

\inputminted[frame=lines,bgcolor=lightgray]{C++}{recursivePow.cpp}

This function will perform \textbf{exactly} $n$ operations, and is therefore $\Theta(n)$ (and thus $O(n)$). However, this function can be improved by performing what is called ``binary exponentiation'' instead. We demonstrate this by evaluating $x^9$: write $9=1001_b$ and observe that\begin{align*}x^9&=x^{1001_b}\\&=x^{1\cdot2^3+0\cdot2^2+0\cdot2^1+1\cdot2^0}\\&=x^{1\cdot2^3}x^{0\cdot2^2}x^{0\cdot2^1}x^{1\cdot2^0}\\&=x^8x^1\end{align*} This creates the far more efficient algorithm:\pagebreak

\inputminted[frame=lines,bgcolor=lightgray]{C++}{binaryPow.cpp}

We see that this version of the function will terminate when the exponent becomes 0, which will occur after integer division by 2 has occurred $\lceil\log(n)\rceil$ times. Thus, this version of the function is $O(\log(n))$.

As a rule of thumb, algorithms which are ``faster'' asymptotically are preferred. So an $O(n)$ algorithm is preferable to an $O(n^2)$ algorithm. This isn't \textit{always} true, however, as an algorithm with an $O(10^{1000}n)$ runtime, while certainly faster \textbf{asymptotically} than an $O(n^2)$ algorithm, is hardly preferable. In fact, the $O(10^{1000}n)$ runtime algorithm will not see such results until $n>10^{1000}$. In other words, it is important to contextualize these bounds.


\section{Queue Operations}
The following queue operations:

enqueue(5), enqueue(3), dequeue(), enqueue(2), enqueue(8), dequeue(), dequeue(), enqueue(9), enqueue(1), dequeue(), enqueue(7), enqueue(6), dequeue(), dequeue(), enqueue(4), dequeue(), dequeue()

appear as follows:

\begin{align*}
\boxed{\phantom{5}}&\xrightarrow{\text{enqueue(5)}}\boxed{5}\xrightarrow{\text{enqueue(3)}}\boxed{3}\boxed{5}\xrightarrow{\text{dequeue()}}\boxed{3}\xrightarrow{\text{enqueue(2)}}\boxed{2}\boxed{3}\xrightarrow{\text{enqueue(8)}}\boxed{8}\boxed{2}\boxed{3}\\&\xrightarrow{\text{dequeue()}}\boxed{8}\boxed{2}\xrightarrow{\text{dequeue()}}\boxed{8}\xrightarrow{\text{enqueue(9)}}\boxed{9}\boxed{8}\xrightarrow{\text{enqueue(1)}}\boxed{1}\boxed{9}\boxed{8}\xrightarrow{\text{dequeue()}}\boxed{1}\boxed{9}\\&\xrightarrow{\text{enqueue(7)}}\boxed{7}\boxed{1}\boxed{9}\xrightarrow{\text{enqueue(6)}}\boxed{6}\boxed{7}\boxed{1}\boxed{9}\xrightarrow{\text{dequeue()}}\boxed{6}\boxed{7}\boxed{1}\xrightarrow{\text{dequeue()}}\boxed{6}\boxed{7}\\&\xrightarrow{\text{enqueue(4)}}\boxed{4}\boxed{6}\boxed{7}\xrightarrow{\text{dequeue()}}\boxed{4}\boxed{6}\xrightarrow{\text{dequeue()}}\boxed{4}
\end{align*}
\end{document}
