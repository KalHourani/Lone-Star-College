\documentclass[a4paper]{article}

%% Language and font encodings
\usepackage[english]{babel}
\usepackage[utf8x]{inputenc}
\usepackage[T1]{fontenc}
\usepackage{listings}
\usepackage{minted}

%% Sets page size and margins
\usepackage[a4paper,top=3cm,bottom=2cm,left=3cm,right=3cm,marginparwidth=1.75cm]{geometry}

%% Useful packages
\usepackage{amsmath}
\usepackage{graphicx}
\usepackage[colorinlistoftodos]{todonotes}
\usepackage[colorlinks=true, allcolors=blue]{hyperref}

\title{List and Iterator ADTs}
\author{Khalid Hourani}

\begin{document}
\maketitle

\section{Reversing an Array}
Reversing an array of $n$ elements involves just swapping elements at certain indices; the element at index 0 must be swapped with the element at index $n-1$, index 1 with index $n-2$, and so on. If $n$ is even, we swap indices $i$ and $n - 1 - i$ until we reach index $\frac{n}{2}$. If $n$ is odd, we do the same until we reach index $\frac{n-1}{2}$. Since $\lfloor{\frac{n}{2}\rfloor}$ is given by  \[\left\lfloor{\frac{n}{2}}\right\rfloor=\begin{cases}\frac{n}{2} & n\text{ is even}\\\frac{n-1}{2} & n\text{ is odd}\end{cases}\] we only need to iterate from index 0 to $\lfloor{\frac{n}{2}\rfloor}$, regardless of the parity of $n$. Luckily, $\lfloor{\frac{n}{2}\rfloor}$ is the same as integer division of $n$ by 2. Thus, we have the following implementation in C++:
\begin{minted}{C++}
void reverse(int* list, int n)
{
	for (int i = 0; i < n / 2; i++)
	{
    	swap(list[i], list[n - 1 - i]);
	}
}
\end{minted}

\section{Randomly Permuting an Array}
To randomly permute an array, we use the \textbf{Fisher-Yates Shuffle} to permute a 0-indexed array: 
\begin{enumerate}
\item Iterate through the array, letting $i$ represent the current index. Begin at 0 and end at $n - 2$. 
\item Choose a random integer $j$ such that $i\leq j \leq n-1$.
\item Swap the $i^{\text{th}}$ and $j^{\text{th}}$ elements of the array. 
\end{enumerate}

In C++, this looks like:
\begin{minted}{C++}
void permute(int* list, int n)
{
	for (int i = 0; i < n - 1; i++)
	{
		int j = randint(i, n - 1); //choose a random integer between i and n - 1
		swap(list[i], list[j]);
	}
}
\end{minted}

\section{Circularly Rotating an Array}
To rotate an array of $n$ elements circularly by $d$ units, swap the elements at indices 0 and $d$, then the elements at indices 0 and $2d$, then 0 and $3d$, and so on, noting that the element at index $kd > n$ has index $kd\bmod{n}$. Repeat until you've performed $n$ swaps. In C++, this looks like:
\begin{minted}{C++}
void rotate(int* list, int n, int d)
{
	d = d % n; //if d < 0 or d >= n, rotating by d is the same as rotating by d mod n
	if (d != 0) //rotating by 0 makes no change to the array
	{
		for (int i = 1; i < n + 1, i++)
		{
			int j = (i * d) % n; //index d, 2d, ...
			swap(list[0], list[j]);
		}
	}
}
\end{minted}
\end{document}
